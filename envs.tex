\documentclass{article}
\usepackage{amsmath, amsthm, amssymb}

% Plain style (bold title, italic text) — used for results
\theoremstyle{plain}
\newtheorem{theorem}{Theorem}[section]
\newtheorem{lemma}[theorem]{Lemma}
\newtheorem{corollary}[theorem]{Corollary}

% Definition style (bold title, upright text) — used for concepts
\theoremstyle{definition}
\newtheorem{definition}[theorem]{Definition}
\newtheorem{example}[theorem]{Example}

% Remark style (italic title, upright text) — used for informal comments
\theoremstyle{remark}
\newtheorem*{remark}{Remark}

\begin{document}

\section{Basic Concepts}

\begin{definition}
A function \( f: \mathbb{R} \to \mathbb{R} \) is said to be \emph{continuous} at a point \( x_0 \in \mathbb{R} \) if
\[
\lim_{x \to x_0} f(x) = f(x_0).
\]
\end{definition}

\begin{theorem}[Extreme Value Theorem]
Let \( f \) be a continuous real-valued function on the closed interval \( [a, b] \). Then \( f \) attains a maximum and a minimum on \( [a, b] \).
\end{theorem}

\begin{proof}
Since \( f \) is continuous on the compact interval \( [a, b] \), by the Weierstrass theorem, \( f \) is bounded and attains its bounds. Therefore, there exist \( x_{\min}, x_{\max} \in [a, b] \) such that
\[
f(x_{\min}) \leq f(x) \leq f(x_{\max}) \quad \text{for all } x \in [a, b].
\]
\end{proof}

\begin{example}
The function \( f(x) = \sin x \) is continuous on \( [0, \pi] \), so it attains its minimum at \( x = \frac{\pi}{2} \) and maximum at \( x = \frac{\pi}{2} \), both equal to 1.
\end{example}

\begin{remark}
Compactness of the domain is essential; the result fails on open intervals.
\end{remark}

\end{document}
