\documentclass{report}

\usepackage[french]{babel}

\usepackage{parskip}
\usepackage{geometry}

\usepackage{amsmath}
\usepackage{amssymb}
\usepackage{amsfonts}
\usepackage{dsfont}
\usepackage{stmaryrd}
\usepackage{esint}

\usepackage{hyperref}

\usepackage[utf8]{inputenc}
\usepackage[T1]{fontenc}
\usepackage{lmodern}

\usepackage{color}
\usepackage{mathdots}
\usepackage{ulem}
\usepackage{cancel}

\usepackage{accents}

\usepackage{url}

\usepackage{pgfplots}
\pgfplotsset{compat=1.15}
\usetikzlibrary{patterns}
\usepackage{mathrsfs}

\usepackage[most]{tcolorbox}
\newtcolorbox{mybox}[1][]{
  enhanced,
  attach boxed title to top left={yshift=-3mm, yshifttext=-1mm, xshift=0.05\textwidth},
  colback=white,         % default background (white for most document classes)
  colframe=black,        % black frame
  colbacktitle=white,    % white background for title area
  coltitle=black,        % black text in title
  fonttitle=\bfseries,   % bold title font
  title=#1
}

\usepackage{subcaption}

\begin{document}

%\textwidth = 690pt

\newgeometry{right = 0in, top = 1.2in}

\url{https://latexref.xyz}

\section*{Symboles} %%%% ------------------------------------------------- SYMBOLES ------------------------------------------------- %%%%

\begin{tabular}{ll}


Exposants $ x^{n} $
&
\texttt{
 x\string^\{n\}
}
\\


Racines $ \sqrt[n]{x} $
&
\texttt{
 \textbackslash{}sqrt[n]\{x\}
}
\\

Limites $ \lim_{x \to +\infty} \quad \displaystyle{\lim_{x \to \infty}} $
&
\texttt{
	\textbackslash{}lim\_\{x \textbackslash{}to +\textbackslash{}infty\} \quad \textbackslash{}displaystyle\{ ... \}
}
\\

$ \xrightarrow[dessous]{dessus} $
&
\texttt{
\textbackslash{}xrightarrow[dessous]\{dessus\}
}
\\

$ \displaystyle{f(x) \longrightarrow_{x \to +\infty}l} $
&
\\

Sommes $ \sum_{k=1}^{n} $
&
\texttt{
\textbackslash{}sum\_\{k=1\}\string^\{n\} \quad \textbackslash{}displaystyle\{ ... \}
}
\\


Produits $ \prod_{i=1}^{n} $
&
\texttt{
\textbackslash{}prod\_\{i=1\}\string^\{n\} \quad \textbackslash{}displaystyle\{ ... \}
}
\\

Union/inter. $ \bigcup_{i=1}^{n} \bigcap_{i=1}^{n} $
&
\texttt{
\textbackslash{}bigcup\_\{i=1\}\string^\{n\} \quad ...cap... \quad \textbackslash{}displaystyle\{ ... \}
}
\\

Bigoplus $ \bigoplus_{k=1}^{n} \bigotimes $
&
\texttt{
\textbackslash{}bigoplus\_\{i=1\}\string^\{n\} \quad \textbackslash{}displaystyle\{ ... \}
}
\\

Intégrales $ \int $
&
\texttt{
\textbackslash{}int
}
\\

Avec bornes $ \int_{a}^{b} blabladx $
&
\texttt{
\textbackslash{}int\_\{a\}\string^\{b\}
}
\\

Avec bornes hautes $ \int \limits_{a}^{b} blabladx $
&
\texttt{
\textbackslash{}int \textbackslash{}limits\_\{a\}\string^\{b\}
}
\\

Intégrales multiples $ \iint_{S} \iiint_{V} $
&
\texttt{
\textbackslash{}iint\_\{S\} \textbackslash{}iiint\_\{V\}
}
\\

Sur un contour fermé $ \oint $
&
\texttt{
\textbackslash{}oint
}
\\

Grandes parenthèses
&
\texttt{
\textbackslash{}left( \textbackslash{}right)
}
\\

\textbf{Autres grands trucs}
&
\texttt{
\textbackslash{}left\textbackslash{}lfloor \textbackslash{}right\textbackslash{}rfloor \textbackslash{}left\textbackslash{}langle \textbackslash{}right\textbackslash{}rangle
}
\\

Valeurs absolues $ |x| \quad \left| x \right| $
&
\texttt{
|x| \quad \textbackslash{}left| x \textbackslash{}right| 
}
\\


Pareil mais chiant $ \mathopen| x \mathclose| $   
&
\texttt{
\textbackslash{}mathopen| x \textbackslash{}mathclose|
}
\\

Egalités $ = \quad \triangleq \quad \ne \quad \equiv $
&
\texttt{
= \textbackslash{}triangleq \textbackslash{}ne ou \textbackslash{}neq \textbackslash{}equiv
}
\\

$ \approx \quad \simeq \quad \sim \quad \cong \quad \thicksim \quad \gtrsim \quad \lesssim $
&
\texttt{
\textbackslash{}approx \textbackslash{}simeq \textbackslash{}sim \textbackslash{}cong \textbackslash{}thicksim \textbackslash{}gtrsim \textbackslash{}lesssim
}
\\


Inégalités $ \leq \quad \geq \quad \leqslant \quad \geqslant $
&
\texttt{
\textbackslash{}leq \textbackslash{}geq \textbackslash{}leqslant \textbackslash{}geqslant
}
\\

$ \ll \quad \gg \quad \prec \quad \succ \quad \preceq \quad \succeq $
&
\texttt{
\textbackslash{}ll \textbackslash{}gg \textbackslash{}prec \textbackslash{}succ \textbackslash{}preceq \textbackslash{}succeq	
}
\\

Ensembles $ \in \quad \notin \quad \subset \quad \not\subset \quad \setminus $
&
\texttt{
\textbackslash{}in \textbackslash{}notin \textbackslash{}subset \textbackslash{}not\textbackslash{}subset \textbackslash{}setminus 
}
\\

$ \subseteq \quad \nsubseteq \quad \cap \quad \cup \quad \sqcup \quad \supset \quad \varnothing $
&
\texttt{
\textbackslash{}(n)subseteq \textbackslash{}c(a/u)p \textbackslash{}sqc(u/a)p \textbackslash{}supset \textbackslash{}varnothing
}
\\

Opérations
&
\\

$ \times \quad \div \quad g \circ f \quad \star \quad \pm \quad \mp \quad \ast $
&
\texttt{
\textbackslash{}times \textbackslash{}div \textbackslash{}circ \textbackslash{}star \textbackslash{}pm \textbackslash{}mp \textbackslash{}ast 
}
\\

$ \wedge \quad \vee \quad \oplus \quad \otimes \quad \ominus \quad \odot $
&
\texttt{
\textbackslash{}wedge \textbackslash{}vee \textbackslash{}oplus \textbackslash{}otimes \textbackslash{}ominus \textbackslash{}odot	
}
\\

Logique
&
\\

$ \forall \quad \exists \quad \exists! \quad \nexists \quad \neg \quad \lor \quad \land $
&
\texttt{
\textbackslash{}forall \textbackslash{}exists \textbackslash{}exists! \textbackslash{}nexists \textbackslash{}neg \textbackslash{}lor \textbackslash{}land
}
\\

$ \implies \quad \iff \quad \impliedby \quad \Rightarrow \quad \Leftrightarrow $
&
\texttt{
\textbackslash{}implies \textbackslash{}iff \textbackslash{}impliedby \textbackslash{}(Right/Leftright)arrow
}
\\

$ \mapsto \quad \longmapsto \quad \to \quad \longrightarrow $
&
\texttt{
\textbackslash{}mapsto \textbackslash{}longmapsto \textbackslash{}to \textbackslash{}longrightarrow
}
\\

Autres flèches
&
\\

$  \nearrow \quad \hookrightarrow \quad \leadsto \quad \rightleftharpoons $
&
\texttt{
\textbackslash{}nearrow \textbackslash{}hookrightarrow \textbackslash{}leadsto \textbackslash{}rightleftharpoons
}
\\

Crochets $ \lfloor \; \; \rfloor \quad \lceil \; \; \rceil \quad [\, \,] \quad \langle \quad \rangle $
&
\texttt{
\textbackslash{}lfloor \textbackslash{}rfloor \textbackslash{}rceil \textbackslash{}lceil [\textbackslash{}, \textbackslash{},] \textbackslash{}langle \textbackslash{}rangle 
}
\\

$\llbracket \quad \rrbracket$
&
\texttt{
\textbackslash{}llbracket \textbackslash{}rrbracket (\textbackslash{}usepackage\{stmaryrd\})
}
\\

Normes $ \lVert \, \cdot \, \rVert \quad |||\cdot||| \quad \| u \|$
&
\texttt{
\textbackslash{}lVert \textbackslash{}rVert \quad |||u||| \quad \textbackslash{}| u \textbackslash{}|
}
\\

\textit{ Ajouter \texttt{ \textbackslash{},} avec \texttt{ \textbackslash{}cdot } }
&
\textit{ et \texttt{ \textbackslash{}\!\! ;} avec des grosses $ \Sigma $ }
\\

Divers / jsp
&
\\

$ \mid \quad \nmid \quad \parallel \quad \bowtie \quad \dagger \quad \ddagger \quad \square	$
&
\texttt{
\textbackslash{}mid \textbackslash{}nmid \textbackslash{}parallel \textbackslash{}bowtie \textbackslash{}dagger \textbackslash{}ddagger \textbackslash{}square
}
\\

$ \triangleleft \quad \triangleright \quad \bigtriangleup \quad \bigtriangledown \quad \Delta \quad \nabla	$
&
\texttt{
\textbackslash{}(big)triangle(left/right) \textbackslash{}Delta \textbackslash{}nabla
}
\\

$ \perp \quad \top \quad \partial \quad \hbar \quad \ell \quad \Re \quad \Im $
&
\texttt{
\textbackslash{}perp \textbackslash{}top \textbackslash{}partial \textbackslash{}hbar \textbackslash{}ell \textbackslash{}Re \textbackslash{}Im
}
\\

$ \epsilon \quad \varepsilon \quad \phi \quad \varphi $
&
\texttt{
\textbackslash{}epsilon \textbackslash{}varepsilon \textbackslash{}phi \textbackslash{}varphi
}
\\


\end{tabular}	

\restoregeometry

\newpage


\section*{Lettres}  %%%% ------------------------------------------------- LETTRES ------------------------------------------------- %%%%


\begin{tabular}{ll}



$ \mathds{N}, \mathds{Z}, \mathds{Q}, \mathds{R}, \mathds{C} $
&
\texttt{
\textbackslash{}mathds\{X\}
}
\\

$ {\rm I\!N}, {\rm I\!R}  $
&
\texttt{
\{\textbackslash{}rm I\textbackslash{}\!\! !X\}
}
\\

$ \mathbb{N}, \mathbb{Z}, \mathbb{Q}, \mathbb{R}, \mathbb{C} $
&
\texttt{
\textbackslash{}mathbb\{X\}
}
\\	

$ \mathbf{N}, \mathbf{Z}, \mathbf{Q}, \mathbf{R}, \mathbf{C} $
&
\texttt{
\textbackslash{}mathbf\{X\}
}
\\

$ \mathcal{ABCDEFHKLMOPSUVW} $
&
\texttt{
\textbackslash{}mathcal\{ABC\}
}
\\

$ \mathfrak{ABCDEFGHIJKLMNOPQRSTUVWXYZ} $
&
\texttt{
\textbackslash{}mathfrak\{ABC\}
}
\\

\end{tabular}


\section*{Accentuations}  %%%% -------------------------------------------- ACCENTUATIONS -------------------------------------------- %%%%


\begin{tabular}{ll}


Accent circonflexe (texte) $ \string^ $
&
\texttt{
\textbackslash{}string\string^
}
\\


Accent circonflexe (accent) \^{a}
&
\texttt{
\textbackslash{}\string^\{a\}
}
\\

Accent circonflexe (accent)(maths) $ \hat{a} $
&
\texttt{
\textbackslash{}hat\{a\}
}
\\

Produit vectoriel / PPCM $ \wedge \vee $
&
\texttt{
\textbackslash{}wedge \textbackslash{}vee
}
\\

Angle genre $ \widehat{abc} $
&
\texttt{
\textbackslash{}widehat\{abc\}
}
\\

Accent aigu $ \acute{a} $
&
\texttt{
\textbackslash{}acute\{a\}
}
\\

(peut s'appliquer à un mot entier)
&
\\

Accent grave $ \grave{a} $
&
\texttt{
\textbackslash{}grave\{a\}
}
\\

Barre $ \bar{a} \quad \bar{10100}^{2} $ (largeur fixe)
&
\texttt{
\textbackslash{}bar\{a\}
}
\\

Overline $ \overline{a} \quad \overline{10100}^{2} $
&
\texttt{
\textbackslash{}overline\{a\}
}
\\

Underline $ \underline{a} \quad \underline{Z} = \underline{U}/\underline{I} $
&
\texttt{
\textbackslash{}underline\{a\}
}
\\

Overbrace $ \overbrace{abc} $
&
\texttt{
\textbackslash{}overbrace\{abc\}
}
\\

Underbrace $ \underbrace{abc} $
&
\texttt{
\textbackslash{}underbrace\{abc\}
}
\\

Overset $ \overset{a}{X} \quad \overset{\circ}{B} $
&
\texttt{
\textbackslash{}overset\{a\}\{X\}
}
\\

Underset $ \underset{a}{X} $
&
\texttt{
\textbackslash{}underset\{a\}\{X\}
}
\\

Text overbrace $ \overbrace{mainthing}^{overtext} $
&
\texttt{
\textbackslash{}overbrace\{mainthing\}\string^\{overtext\}
}
\\

Text underbrace $ \underbrace{mainthing}_{undertext} $
&
\texttt{
\textbackslash{}underbrace\{mainthing\}\_\{undertext\}
}
\\

Point $ \dot{x} $
&
\texttt{
\textbackslash{}dot\{x\}
}
\\

Point point $ \ddot{x} $
&
\texttt{
\textbackslash{}ddot\{x\}
}
\\

Tilde $ \tilde{u} $
&
\texttt{
\textbackslash{}tilde\{u\}
}
\\


Widetilde $ \widetilde{abc} $
&
\texttt{
\textbackslash{}widetide\{abc\}
}
\\

Vecteur $ \overrightarrow{v} \quad \overrightarrow{grad} $ 
&
\texttt{
\textbackslash{}overrightarrow\{grad\}
}
\\

Vecteur $ \vec{v} \quad \vec{grad} $  (moche)
&
\texttt{
\textbackslash{}vec\{v\}
}
\\

Produit scalaire
&
\\

$ \overrightarrow{u} \cdot \overrightarrow{v} $
&
\texttt{
\textbackslash{}cdot
}
\\

$ \overrightarrow{u} \cdotp \overrightarrow{v} $
&
\texttt{
\textbackslash{}cdotp
}
\\

$ \overrightarrow{u} \bullet \overrightarrow{v} $
&
\texttt{
\textbackslash{}bullet
}
\\

$ \accentset{\circ}{A} $
&
\texttt{
\textbackslash{}accentset\{\textbackslash{}circ\}\{I\} \quad \textbackslash{}usepackage\{accents\}
}
\\




%
&
\texttt{
%\textbackslash{}
}
\\

%
&
\texttt{
%\textbackslash{}
}
\\

\end{tabular}



\subsection*{Espacements} % \normalfont (+ alignement)

%\begin{align*}

\begin{tabular}{ll}

$ x^{2} \! + \! 3x \! + \! 2 $
& 
\texttt{ \textbackslash{}\!\! ! }
\\

$ x^{2}+3x+2 $ 
&
\texttt{ [rien] }
\\

$ x^{2} + 3x + 2 $ 
&
\texttt{ [espaces] }
\\

$ x^{2}\, +\, 3x\, +\, 2 $
&
\texttt{ \textbackslash{},}
\\

$ x^{2}\: +\: 3x\: +\: 2 $
&
\texttt{ \textbackslash{}\!\!\! :}
\\

$ x^{2}\; +\; 3x\; +\; 2 $
&
\texttt{ \textbackslash{}\!\! ;}
\\

$ x^{2}\ +\ 3x\ +\ 2 $
&
\texttt{ \textbackslash{}}
\\

$ x^{2}\quad +\quad 3x\quad +\quad 2 $
&
\texttt{ \textbackslash{}quad}
\\

$ x^{2}\qquad +\qquad 3x\qquad +\qquad 2 $
&
\texttt{ \textbackslash{}qquad}
\\

\end{tabular}

%\end{align*} (aligne devant les & (à ajouter du coup, comme point d'ancrage)

\subsection*{Fractions avec \normalfont \texttt{ \textbackslash{}frac\{\}\{\}  } }

\begin{center}
$ 1 + \frac{1}{1+\frac{1}{1+...}} $
\qquad \qquad
$ 1 + \frac{1}{1+\frac{1}{1+\frac{1}{1+\frac{1}{1+\frac{1}{1+...}}}}} $
\end{center}

\subsection*{Fractions avec \normalfont \texttt{ \textbackslash{}cfrac\{\}\{\}  } }

\begin{center}
$$ 1 + \cfrac{1}{1+\cfrac{1}{1+...}},  
\qquad\qquad
1 + \cfrac{1}{1+\cfrac{1}{1+\cfrac{1}{1+\cfrac{1}{1+\cfrac{1}{1+...}}}}} $$
\end{center}

\subsection*{Autres}

\subsubsection*{Fonction usuelles}

\begin{center}

$ \cos(x), \sin(x), \tan(x), \arccos(x), \arcsin(x), \arctan(x), \cosh(x), \sinh(x), $ \\ 
$ \tanh(x), \cosh^{-1}(x), \sinh^{-1}(x), \tanh^{-1}(x),  \exp(x), \ln(x), \log(x), \log_{b}(a) $ \\
$ \arg(x), \dim(x), \min(a,b), \max(a,b), \gcd(a,b) $ \\

\bigskip

$ cos(x), sin(x), tan(x), arccos(x), arcsin(x), arctan(x), cosh(x), sinh(x), $ \\
$ tanh(x), cosh^{-1}(x), sinh^{-1}(x), tanh^{-1}(x),  exp(x), ln(x), log(x), log_{b}(a) $ \\
$ arg(x), dim(x), min(a,b), max(a,b), gcd(a,b) $ \\

\end{center}

\textbf{Pour d'autres fonctions : }
\texttt{ \$ \textbackslash{}mathrm\{PGCD\} \$}

\textbf{Pour ajouter des trucs en dessous comme ça :}
$$ \sum_{\substack{(x,K) \, tq \\ x \in \Omega \\ K \subset I_{x}}} (-1)^{\mathrm{Card(K)}} $$ 
\begin{center}  \texttt{ \textbackslash{}sum\_\{ \textbackslash{}substack\{ ... \textbackslash{}\textbackslash{}
... \textbackslash{}\textbackslash{} ... \} \}} \\
\textbf{Ne pas oublier les brackets pour substack.}\end{center}

\newpage

\begin{center} 

$$ \iint_{S} \mu(x,y) dxdy $$

$$ \int \!\!\!\!\!\! \bigcirc \!\!\!\!\!\! \int_{\Sigma} $$

\texttt{
\$\$ \textbackslash{}int \textbackslash{}!\textbackslash{}!\textbackslash{}!\textbackslash{}!\textbackslash{}!\textbackslash{}!\textbackslash{}! \textbackslash{}bigcirc \textbackslash{}!\textbackslash{}!\textbackslash{}!\textbackslash{}!\textbackslash{}!\textbackslash{}!\textbackslash{}! \textbackslash{}int\_\{Sigma\} \$\$
}

$$ \left\lVert \: \sum_{i=1}^{n} \lambda_i e_i \: \right\rVert $$ 

\texttt{
\textbackslash{}left\textbackslash{}lVert \textbackslash{}\!\!\!: \textbackslash{}sum ... \textbackslash{}\!\!\!: \textbackslash{}right\textbackslash{}rVert
}


$$ f(x) \to \ell $$ 
\texttt{ \$ f(x) \textbackslash{}to \textbackslash{}ell \$} 

$$ f(x) \rlap{\hskip 3mm $/$} \longrightarrow \ell $$ 
\texttt{ \$ f(x) \textbackslash{}rlap\{\textbackslash{}hskip 3mm \$/\$\} \textbackslash{}longrightarrow \textbackslash{}ell \$}

\end{center}

\bigskip
\bigskip
\bigskip

\Large \textbf{Changer la numérotation des part, chapter, section, etc}

\normalsize

Déjà, dans l'ordre

\texttt{ \textbackslash{}part \\ \textbackslash{}chapter \\ \textbackslash{}section \\ \textbackslash{}subsection \\ \textbackslash{}subsubsection \\ \textbackslash{}paragraph \\ \textbackslash{}subparagraph }

\chapter{Chapter}

\section{Section}

\subsection{Subsection}

\subsubsection{Subsubsection}

\paragraph{Paragraph}

\subparagraph{Subparagraph}

\paragraph{}

\paragraph{}

Et ensuite pour renommer

\texttt{ 
\textbackslash{}renewcommand\textbackslash{}thepart\{\textbackslash{}arabic\{part\}\} \\
\textbackslash{}renewcommand\textbackslash{}thesection\{\textbackslash{}arabic\{section\}\} \\
\textbackslash{}renewcommand\textbackslash{}thesubsection\{\textbackslash{}arabic\{subsection\}\} \\
\textbackslash{}arabic : 1, 2, 3, ... \\
\textbackslash{}alph : a, b, c, ... \\
\textbackslash{}Alph : A, B, C, ... \\
\textbackslash{}roman : i, ii, iii, ... \\
\textbackslash{}Roman : I, II, III, ...
}	

\newpage

\textbf{Systèmes d'équations}

\begin{align*}
x + y + z &= a\\
x - y &= b\\
z &= c
\end{align*}

\begin{center}
\texttt{
\textbackslash{}begin\{align*\} .. \&= .. \textbackslash{}\textbackslash{} .. \&= .. \textbackslash{}end\{align*\}
}\\
\end{center}

\begin{eqnarray*}
x + y + z &=& a\\
x - y &=& b\\
z &=& c
\end{eqnarray*}

\begin{center}
\texttt{
\textbackslash{}begin\{eqnarray*\} .. \&=\& .. \textbackslash{}\textbackslash{} .. \&=\& .. \textbackslash{}end\{eqnarray*\}
}\\
\end{center}

\textbf{Une seule esperluette pour align, deux pour  array.}\\
Enlever les astérisques numérote les équations. 

$$
\left\{
\begin{array}{r c l}
x + y + z &=& a\\
x - y &=& b\\
z &=& c
\end{array}
\right.
$$

\texttt{
\$\$ \\
\textbackslash{}left\textbackslash{}\{ \\
\textbackslash{}begin\{array\}\{r c l\} \\
... \&=\& ... \\
... \&=\& ... \\
\textbackslash{}end\{array\} \\
\textbackslash{}right. \\
\$\$
} \\
Le point représente une \textbf{absence de délimiteurs.} \\
(on pourrait choisir de refermer l'accolade à droite).
\bigskip
\textbf{Autres délimiteurs :}





$$
\left(
\begin{array}{r c l}
x + y + z &=& a\\
y - z &=& b\\
z &=& c 
\end{array}
\right)
\left[
\begin{array}{r c l}
x + y + z &=& a\\
y - z &=& b\\
z &=& c 	
\end{array}
\right]
$$
$$
\left|
\begin{array}{r c l}
x + y + z &=& a\\
x - y &=& b\\
z &=& c
\end{array}
\right|
\quad
\left\|
\begin{array}{r c l}
x + y + z &=& a\\
x - y &=& b\\
z &=& c
\end{array}
\right\|
$$
\bigskip
\begin{center}
\texttt{ \textbackslash{}left( \qquad \textbackslash{}left[ \qquad \textbackslash{}left| \qquad \textbackslash{}left\textbackslash{}| }
\end{center}

\newpage

\subsubsection*{Les matrices}

Avec l'environnement \texttt{ array}  (et des \textbackslash{}quad) :
$$
\left[
\begin{array}{r c l}
1 \quad 2 \quad 3 \\
4 \quad 5 \quad 6 \\
7 \quad 8 \quad 9
\end{array}
\right] $$

\texttt{
\$\$ \\
\textbackslash{}left[ \\
\textbackslash{}begin\{array\}\{r c l\} \\
1 \textbackslash{}quad 2 \textbackslash{}quad 3 \textbackslash{}\textbackslash{} \\
4 \textbackslash{}quad 5 \textbackslash{}quad 6 \textbackslash{}\textbackslash{} \\
7 \textbackslash{}quad 8 \textbackslash{}quad 9 \\
\textbackslash{}end\{array\} \\
\textbackslash{}right] \\
\$\$ 
}

Avec \texttt{ matrix, pmatrix, bmatrix, vmatrix, 	Bmatrix, Vmatrix} :

$
\begin{matrix}
1 & 2 & 3 \\
4 & 5 & 6 \\
7 & 8 & 9
\end{matrix}
$
\quad
$
\begin{pmatrix}
1 & 2 & 3 \\
4 & 5 & 6 \\
7 & 8 & 9
\end{pmatrix}
$
\quad
$
\begin{bmatrix}
1 & 2 & 3 \\
4 & 5 & 6 \\
7 & 8 & 9
\end{bmatrix}
$
\quad
$
\begin{vmatrix}
1 & 2 & 3 \\
4 & 5 & 6 \\
7 & 8 & 9
\end{vmatrix}
$
\quad
$
\begin{Bmatrix}
1 & 2 & 3 \\
4 & 5 & 6 \\
7 & 8 & 9
\end{Bmatrix}
$
\quad
$
\begin{Vmatrix}
1 & 2 & 3 \\
4 & 5 & 6 \\
7 & 8 & 9
\end{Vmatrix}
$

\texttt{
\$\$ \\
\textbackslash{}begin\{matrix\} \\
.. \& .. \& .. \textbackslash{}\textbackslash{} \\
.. \& .. \& .. \textbackslash{}\textbackslash{} \\
.. \& .. \& .. \\
\textbackslash{}end\{matrix\} \\
\$\$
} 
\bigskip

\textbf{Pour les mettres les unes à côté des autres :} utiliser des \textbf{monodollars}, une paire par matrice, éventuellement séparés par des \texttt{ \textbackslash{}quad} comme ici.

\textbf{Technique ultime utilisée par Benhamou Sensei :}

$$ \left( \: \begin{matrix} 1 & 2 & 3 \\ 4 & 5 & 6 \\ 7 & 8 & 9 \end{matrix} \: \right) $$

Entourer l'environnement \texttt{ matrix} par \texttt{ \textbackslash{}left( \textbackslash{}\!\!\!:} et 
\texttt{ \textbackslash{}\!\!\!: \textbackslash{}right) }.

\textbf{Pointillés :}

\begin{tabular}{lllll}
$ \cdots $ & $ \ldots $ & $ \vdots $ & $ \ddots $ & $ \iddots $ \\
\texttt{ \textbackslash{}cdots } & \texttt{ \textbackslash{}ldots } & \texttt{ \textbackslash{}vdots } & \texttt{ \textbackslash{}ddots} & \texttt{ \textbackslash{}iddots} (\texttt{ \textbackslash{}usepackage\{mathdots\}})
\end{tabular}

La commande \texttt{ \textbackslash{}phantom} permet de gérer les alignements et le centrage des nombres dans chaque case.


%$
%\begin{bmatrix}
%   1 & 12345 & 3 \\
%   94 & 5 & -6 \\
%   7 & 8 & 9 
%\end{bmatrix}
%$
%\quad
%$
%\begin{bmatrix}
%   \phantom{9}1 & 12345 & \phantom{-}3 \\
%   94 & \phantom{1234}5 & -6 \\
%   \phantom{9}7 & \phantom{1234}8 & \phantom{-}9 
%\end{bmatrix}
%$

\subsection*{Matrices et applications}

Faire une belle application (Aymeric sensei no jutsu)

$$
\begin{array}{ccccc} 
\phi & : & \mathbb{N}^* & \to & \mathbb{N} \\ 
& & n & \mapsto & Card \left\{ k \in |[1,n]|, \, pgcd(k,n) = 1 \right\} \\ 
\end{array}
$$


\texttt{
\$ \textbackslash{}begin\{array\}\{ccccc\}  \\
f \& : \& \{\} \& \textbackslash{}to \& \{\} \textbackslash{}\textbackslash{}  \\
\& \& x \& \textbackslash{}mapsto \& ... \textbackslash{}\textbackslash{}  \\
\textbackslash{}end\{array\} \$
}

\textbf{On retiendra} : 
\texttt{ \textbackslash{}begin\{array\}\{ccccc\} } \\
Et mettre des esperluettes entre chaque truc, deux au début de la deuxième ligne (pour aligner $f$)
et pas en fin de ligne

Faire des belles matrices (Aymeric sensei no jutsu)

$$
\begin{bmatrix} 
a & b & c \\
d & e & f \\
g & h & i
\end{bmatrix}
$$

\texttt{
\$ \textbackslash{}begin\{bmatrix\}  \\
a \& b \& c \textbackslash{}\textbackslash{}  \\
a \& b \& c \textbackslash{}\textbackslash{}  \\
a \& b \& c \textbackslash{}\textbackslash{}  \\
\textbackslash{}end\{bmatrix\} \$
}

\textbf{On retiendra} : \texttt{ array bmatrix }, \\
esperluettes entre les objets, \\
et on revient à la ligne avec \texttt{ \textbackslash{}\textbackslash{} } 

%\textbackslash{} begin\{array\}\{ccccc\} 
%f \& : \& \{\} \& \to \& ... \textbackslash \textbackslash
%\& \& x \mapsto \& ...\textbackslash \textbackslash
%\textbackslash end\{array\}

%}

Il faut deux esperluettes au début de la deuxième ligne \\
pour que le f soit un peu décalé vers la gauche \\
Et sinon, un entre chaque truc, sauf en fin de ligne (y'a plus rien à aligner) 

Matrices par blocs

$$
\left(
\begin{array}{c|c}
A & B\\
\hline
C & D
\end{array}
\right)
$$

\texttt{
\$ \textbackslash{}left( \textbackslash{}begin\{array\}\{c|c\} \\
A \& B \textbackslash{}\textbackslash{} \\
\textbackslash{}hline \\
C \& D \\
\textbackslash{}end\{array\} \textbackslash{}right) \$
} Attention aux \textbackslash{}\textbackslash{}

%%%%%%%%% POINTILLES MATRICES   ---------    https://tex.stackexchange.com/questions/32217/3-dots-in-matrix/32221 %%%%%%%

\subsection*{Autres}

\textbf{Saut de ligne} : \texttt{ \textbackslash{}bigskip} \quad (ne \textit{pas} mettre en fin de ligne) \\
\textbf{\textcolor{red}{Couleurs :}} \texttt{ \textbackslash{}textcolor\{couleur\}\{texte\}  (\textbackslash{}usepackage\{color\})}\\
Pour afficher \texttt{ \textbackslash{}\!\! !}, il faut ajouter \texttt{ \textbackslash{}\!\! !\textbackslash{}\!\! !} entre \texttt{ \textbackslash{}} et \texttt{ !}. \\
Pour afficher \texttt{ \textbackslash{}\!\!\!:}, il faut ajouter \texttt{ \textbackslash{}\!\! !\textbackslash{}\!\! !\textbackslash{}\!\! !} entre \texttt{ \textbackslash{}} et \texttt{ :}. \\
Pour élargir le corps du texte : \texttt{ \textbackslash{}usepackage\{geometry\}} \quad \texttt{ \textbackslash{}textwidth = 426pt} \textbf{(non)} \\
Pour afficher certains caractères hors du math mode : \texttt{ \textbackslash{}textbullet \textbackslash{}dagger \textbackslash{}textasteriskcentered (p3 de la Big Great List)} \\
Pour encadrer : \texttt{ \textbackslash{}fbox\{...\}} ; remettre impérativement des dollars à l'intérieur si c'est dans une équation. \\
Lettre grecques en math mode \textit{uniquement}.\\
\texttt{ \textbackslash{}displaystyle\{\} :} mettre les dollars à \textbf{l'extérieur}. 

\subsubsection*{Remarques très anecdotiques}

Il y a  \texttt{ :} et \texttt{ colon} \\
Autres façons de faire l'ensemble vide : \texttt{ \textbackslash{}emptyset \textbackslash{}o } \\
$ \diamond $ \quad \texttt{ \textbackslash{}diamond} $ \surd $   \texttt{ \textbackslash{}surd}

\texttt{ \textbackslash{}Arrowvert}


\subsubsection*{Import de packages}

Belles lettres : \\
\texttt{
\textbackslash{}usepackage[utf8]\{inputenc\} \\
\textbackslash{}usepackage[T1]\{fontenc\} \\
\textbackslash{}usepackage\{lmodern\}
} \\
Français : \texttt{ \textbackslash{}usepackage[french]\{babel\} } \\
Quelques symboles de maths : \texttt{ \textbackslash{}usepackage\{amsmath\} } \\
Plus de symboles de maths : \texttt{ \textbackslash{}usepackage\{amssymb\} } \\
Belles lettres de maths : \texttt{ \textbackslash{}usepackage\{dsfont\} } (sans s) \\
Quelques polices : \texttt{ \textbackslash{}usepackage\{amsfonts\} } (avec s)\\
En particulier, cette police-la : \texttt{ \textbackslash texttt \{ ... \}}\\
\textsc{Small caps: } \texttt{ \textbackslash textsc\{Small caps\} }

\textbf{Symboles utiles}

\begin{verbatim}
\newcommand{\og}{«\:}
\newcommand{\fg}{»\:}
\end{verbatim}

These are provided by default in $\texttt{\textbackslash usepackage[french]{babel}}$

\newpage

\section*{Commandes}

\begin{verbatim}
\newcommand{\cmdName}[numArgs][defaultValue]{substitution}
\renewcommand{\mycommand}{Goodbye, World!}
\end{verbatim}

\subsection*{Environments}

\begin{verbatim}
\usepackage{amsthm}

% Plain style : bold title, italic text
\theoremstyle{plain}
\newtheorem{theorem}{Theorem}[section] 
% "theorem" refers to the fact that 
% \begin{theorem} ... \end{theorem} must be used.
% "Theorem" is what will be written at the beginning.
% [section] here indicated that the counter will be reset at each \section.
\newtheorem{lemma}[theorem]{Lemma}
% [theorem] here indicates that the counter for lemmas 
% will be shared with the theorem's.
\newtheorem{corollary}[theorem]{Corollary}
% idem.

% Definition style : bold title, upright text (=normal, non italic)
\theoremstyle{definition}
\newtheorem{definition}[theorem]{Definition}
\newtheorem{example}[theorem]{Example}

% Remark style : italic title, upright text (=normal, non italic)
\theoremstyle{remark}
\newtheorem*{remark}{Remark}
% Use the starred version to have all remarks unnumbered. 

\begin{document}

\section{Basic Concepts}

\begin{definition} ... \end{definition}

\begin{theorem}[Extreme Value Theorem] ... \end{proof}
% Theorem name

\begin{proof} ... \end{proof}
% The proof environment is provided by default. 
% It adds "Proof. ... \qed".
\end{document}

\end{verbatim}

To get section-wise numbering : 

\begin{verbatim}
\numberwithin{definition}{section}
\numberwithin{theorem}{section}
\numberwithin{corollary}{section}
\numberwithin{proposition}{section}
\numberwithin{notation}{section}
\numberwithin{remark}{section}
\numberwithin{hypothesis}{section}
\end{verbatim}

\subsection*{Include other files}

\begin{verbatim}
  \input{filename} % Directly includes code, does not require preamble
  \include{filename} % Includes code in new page, does not require preamble
\end{verbatim}

\subsection*{Indent paragraphs}

(suitable for books)

\begin{verbatim}
\usepackage{parskip}
\usepackage{indentfirst}
\setlength{\parindent}{15pt} 
\end{verbatim}

\subsection*{Tableaux}

\begin{tabular}{ccccc}
a & b \\ 
c & d 
\end{tabular}

\begin{verbatim}
\begin{tabular}{ccccc}
a & b \\ 
c & d 
\end{tabular}
\end{verbatim}

\begin{table}[ht]
  \centering
  \caption{Multi-column alignment}
  \label{tab:align}
  \begin{tabular}{lcr}
    Left & Center & Right \\
    \hline
    A    & B      & C \\
    D    & E      & F \\
  \end{tabular}
\end{table}

\begin{verbatim}
\begin{table}[ht]
  \centering
  \caption{Multi-column alignment}
  \label{tab:align}
  \begin{tabular}{lcr}
    Left & Center & Right \\
    \hline
    A    & B      & C \\
    D    & E      & F \\
  \end{tabular}
\end{table}
\end{verbatim}

\subsection*{Cross-references}

\begin{verbatim}
\hyperref[labelname]{custom text}
\end{verbatim}

\begin{verbatim}
	\section{Introduction} \label{sec:intro}
	As seen in Section~\ref{sec:intro}, ...

	\begin{equation} \label{eq:newton}
	F = ma
	\end{equation}
	See Equation~\eqref{eq:newton}.

	\begin{figure}
	\includegraphics{image.png}
	\caption{Sample Image}
	\label{fig:sample}
	\end{figure}
	Refer to Figure~\ref{fig:sample}.

	\begin{table}
	\caption{Sample Data}
	\label{tab:data}
	\end{table}
	Refer to Table~\ref{tab:data}.

	\begin{theorem} \label{thm:main}
	This is a theorem.
	\end{theorem}
	Theorem~\ref{thm:main}

	See page~\pageref{fig:sample}. % Can be any

	\begin{enumerate}
	\item First item \label{item:first}
	\end{enumerate}
	See item~\ref{item:first}.	
\end{verbatim}

\textbf{Note :} The \texttt{\~} are non-breaking spaces. This allows for \LaTeX to never break to a newline.

\subsection*{Number and label equations}

\begin{equation}\label{eq:main-result-ginibre}
    \mathcal W_{KR} (\mathfrak S^R, \mathfrak S^R_N ) \leqslant \sqrt{\frac 2 \pi } R e^{-c^2} 
\end{equation}

The equation~\ref{eq:main-result-ginibre} 

The equation~\eqref{eq:main-result-ginibre} 

\begin{verbatim}
  \usepackage{hyperref}

  \begin{equation}\label{eq:main-result-ginibre}
    \mathcal W_{KR} (\mathfrak S^R, \mathfrak S^R_N ) \leqslant \sqrt{\frac 2 \pi } R e^{-c^2} 
  \end{equation}

  The equation~\ref{eq:main-result-ginibre} 

  The equation~\eqref{eq:main-result-ginibre} 
\end{verbatim}

Just import \texttt{hyperref} to get 

\subsection*{Align equations}

\begin{align*}
  x &= y \\
  y &\leqslant z
\end{align*}

\begin{verbatim}
  \begin{align*}
    x &= y \\
    y &\leqslant z
  \end{align*}
\end{verbatim}

Use the unstarred version \texttt{\textbackslash begin\{align\}} to number the equation.

\texttt{align} is used without math equations. It's a substitute (or another version) for math equations.

\texttt{aligned} is a version to be used inside math equations.

\[
\forall n \in \mathbf N, 
\left\{
  \begin{aligned}
    a_{n+1} &\in (b_n, 1) \\
    b_{n+1} &= \sqrt{ \frac{a_{n+1}^2 + u_n(1-a_{n+1}^2)}{a_{n+1}^2 + (1+u_n)(1-a_{n+1}^2)} }
  \end{aligned}
\right.
\]

\begin{verbatim}
\[
  \forall n \in \mathbf N, 
  \left\{
    \begin{aligned}
      a_{n+1} &\in (b_n, 1) \\
      b_{n+1} &= \sqrt{ \frac{a_{n+1}^2 + u_n(1-a_{n+1}^2)}{a_{n+1}^2 + (1+u_n)(1-a_{n+1}^2)} }
    \end{aligned}
  \right.
  \]
\end{verbatim}

\subsection*{Multiline equations}

\begin{verbatim}
  \begin{multline*}
    long loooooong equation
  \end{multline*}
\end{verbatim}

No math mode or equation needed. 

Use starred version if you don't want it to be numbered. 

Remove the star if you want it to be numbered.

Another precision : 

\begin{verbatim}
  \begin{unnumberredenvironment*}
  \end{unnumberredenvironment*}
  // Star at the END 

  \begin*{section}
  \end*{section}

  \begin*{subsection}
  \end*{subsection}
  // Star in the MIDDLE
\end{verbatim}

\subsection*{Bibliography}

In a \texttt{.bib} file : 

\begin{verbatim}
@book{PPIntro2003,
  author={D. J. Daley and D. Vere-Jones},
  title={An Introduction to the Theory of Point Processes},
  journal={Probability and Its Applications},
  year={2003},
  volume={1},
  edition={2},
  doi={https://doi.org/10.1007/b97277},
  publisher={Springer},
  isbn={978-0-387-95541-4},
  isbn_softcover={978-1-4757-8109-0},
  isbn_ebook={978-0-387-21564-8},
}

@incollection{Decreusefond2016,
  title={Determinantal point processes},
  author={Decreusefond, Laurent, ...},
  booktitle={Stochastic analysis, ...},
  pages={311--342},
  year={2016},
  publisher={Springer}
}

@misc{ParisEst2022,
 author = {Nathael Gozlan, ...},
 title = {Notes de cours sur le Transport Optimal},
 year = {2022},
}

@misc{MorozSoftware,
  author       = {G. Moroz},
  title        = {Determinantal point process},
  year         = {2020},
  howpublished = {\url{...}},
  note         = {Software, doi:10.5281/zenodo.4088585  },
}
\end{verbatim}

And then, 

\begin{verbatim}
\usepackage{cite}

\cite{ParisEst2022}

...

\bibliographystyle{plain}
\bibliography{Article, books, misc} % Files *.bib

$ > pdflatex main.tex ; bibtex main ; pdflatex main.tex ; pdflatex main.tex
\end{verbatim}

Use \texttt{bibtool -s -i misc.bib -o misc.bib -r 'author,year,title'} to sort and reformat your \texttt{*.bib} files !

Bibliography styles :
\begin{itemize}
\item \texttt{plain}: Entries are listed in alphabetical order by author and are labeled numerically.
\item \texttt{unsrt}: Similar to plain, but entries are listed in the order they are cited in the document.
\item \texttt{alpha}: Entries are listed alphabetically and are labeled with an alphabetic code based on the author's name and the year of publication.
\item \texttt{abbrv}: Similar to plain, but uses abbreviated journal names and first names of authors.
\end{itemize}

\subsection*{Minipages}

\begin{minipage}{0.9\textwidth}
    This is a block of text that takes 90\% of the page text width block.
\end{minipage}

\begin{verbatim}
\begin{minipage}{0.9\textwidth}
    This is a block of text that takes 90\% of the page text width block.
\end{minipage}
\end{verbatim}

\begin{verbatim}
\setlength{\fboxsep}{5pt}
\setlength{\fboxrule}{1pt}
\fbox{
  \begin{minipage}{\dimexpr\textwidth-2\fboxsep-2\fboxrule}
  This text fits perfectly inside a box that is the full width of the text area.
  \end{minipage}
}
\end{verbatim}

\textbf{Note :} \texttt{\textbackslash dimexpr -2\textbackslash fboxsep-2\textbackslash fboxrule} computes the width of the minipage.

\texttt{\textbackslash dimexpr a - b + c } allows to perform computations (in particular provides the syntax for it)

textbackslash \texttt{\textbackslash-2\textbackslash fboxsep-2\ fboxrule} Computes the text block width, minus twice the fbox padding, minus twice the fbox line width.

\texttt{\textbackslash parbox} and \texttt{\textbackslash makebox}

\texttt{\textbackslash noindent} allows to remove indentation if needed

\subsection*{More fancy boxes}

\begin{mybox}[Title]
  Hi mom
\end{mybox}

\begin{verbatim}
\usepackage[most]{tcolorbox}

\newtcolorbox{mybox}[1][]{
  enhanced,
  attach boxed title to top left={yshift=-3mm, yshifttext=-1mm, xshift=0.05\textwidth},
  colback=white,         % default background (white for most document classes)
  colframe=black,        % black frame
  colbacktitle=white,    % white background for title area
  coltitle=black,        % black text in title
  fonttitle=\bfseries,   % bold title font
  title=#1
}

\begin{mybox}[Title]
  Hi mom
\end{mybox}
\end{verbatim}

\textbf{Note :} Do not put two carriage return between two arguments (it becomes interpreted as a paragraph end and doen't work anymore)

\section*{New environment}

\texttt{\textbackslash newenvironment\{name\}[number of arguments][default value]\{begin definition\}\{end definition\}}

\section*{Sections numbering}

\texttt{Chatper.Section.Subsection.Subsubsection} numbering :

\begin{verbatim}
  \texttt{\renewcommand\thesubsubsection{\thesubsection.\alph{subsubsection})} 
  % ^ 2.1.4.a)
  \setcounter{secnumdepth}{3}
  % ^ Required in the book documentclass, subsubsections are 
  % unnumbered by default in book documentclass
\end{verbatim}

\texttt{Subsubsection} numbering :

\begin{verbatim}
  \texttt{\renewcommand\thesubsubsection{\alph{subsubsection})} 
  % ^ a)
  \setcounter{secnumdepth}{3} 
  % ^ Required in the book documentclass, subsubsections are 
  % unnumbered by default in book documentclass
\end{verbatim}

\section*{Table of contents}

\texttt{\textbackslash tableofcontents}, place in \texttt{document} wherever you want you table of contents to be (at the begining, at the end, before/after introduction...)

\textbf{Table of contents numbering depth}

\begin{verbatim}
\setcounter{tocdepth}{0}
% ^ In book documentclass : show Parts, chapters
\setcounter{tocdepth}{1}
% ^ In book documentclass : show Parts, chapters, sections
\setcounter{tocdepth}{2}
% ^ In book documentclass : show Parts, chapters, sections, subsections 
\setcounter{tocdepth}{3}
% ^ In book documentclass : show Parts, chapters, sections, subsections, subsubsections
\end{verbatim}

\section*{Packages}

\subsection*{Espacements}

\begin{verbatim}
\usepackage{parskip}
\includegraphics[scale=0.5]{img.jpg}
\end{verbatim}

\subsection*{Unités pour la longueur}

\texttt{pt} (72.25 \texttt{pt} = 1 \texttt{in})

\texttt{in} (1 \texttt{in} = 2.54 \texttt{cm})

\texttt{cm} (\texttt{cm})

\texttt{mm} (\texttt{mm})

\subsection*{\texttt{graphicx}}

\begin{verbatim}
\usepackage{graphicx}
\includegraphics[scale=0.5]{img.jpg}
\end{verbatim}

\subsection*{Fancy headers}

\begin{verbatim}
\renewcommand{\chaptermark}[1]{
	\markboth{\chaptername\ \thechapter.\ #1}{}
} % This actually sets \leftmark
\renewcommand{\sectionmark}[1]{
	\markright{\thesection.\ #1}
} % This sets \rightmark
\pagestyle{fancy}
\fancyhead[LE,RO]{\nouppercase{\rightmark}} % O : for odd pages
\fancyhead[LO,RE]{\nouppercase{\leftmark}} % E : for even pages
\end{verbatim}

\textbf{Note :} If \texttt{\textbackslash documentclass[oneside]\{book\}} is set, then the E and O are useless (because there's only one sode)

\textbf{Note : Make sure to deal with \texttt{fancyhdr} AFTER setting the \texttt{geometry} settings !}

Otherwise, some things such as \texttt{\textbackslash setlength\{\textbackslash headheight\}\{15pt\}} don't work.

\subsection*{Algorithms}

\begin{verbatim}
\begin{algorithm}
    \caption{This algorithm does ...}
    \begin{algorithmic}[1]
    \State \textbf{Input :} ...
    \State \textbf{Output :} ...
    \State $x \leftarrow ... $
    \State ...
    \For{$i \leftarrow 2$ to n}
        \State ...
    \EndFor
    \end{algorithmic}
\end{algorithm}
\end{verbatim}

\texttt{\textbackslash begin\{algorithmic\}[1]} Sets line numbering.

\texttt{\textbackslash begin\{algorithmic\}[2]} Would set line numbering, but numbering every two line.

\texttt{\textbackslash begin\{algorithmic\}} Disables line numbering.

\newpage

\subsection*{Premable for the article}

\begin{verbatim}

\documentclass[11pt]{article}
%
\usepackage[utf8]{inputenc}
\usepackage[T1]{fontenc}
\usepackage{lmodern}
\usepackage[english]{babel}
\newcommand{\og}{«\:}
\newcommand{\fg}{»\:}
%
\usepackage{amsmath}
\usepackage{amssymb}
\usepackage{amsfonts}
\usepackage{amsthm}
%
\usepackage{svg}
\usepackage{graphicx}
\usepackage{tikz}
%
\usepackage{cite}
\usepackage{hyperref}
\usepackage{geometry}
%
\setlength{\parindent}{15pt} % or any desired value
\setlength{\parskip}{6pt}    % optional: 
\newgeometry{left = 1in, right = 1in, top = 1in}
%
\usepackage{algorithm}
\usepackage{algpseudocode} % For algorithmic environment
%
\theoremstyle{definition}
\newtheorem{definition}{Definition}
\newtheorem{theorem}[definition]{Theorem}
\newtheorem{corollary}[definition]{Corollary}
\newtheorem{notation}[definition]{Notation}
\newtheorem{proposition}[definition]{Proposition}
\newtheorem{remark}[definition]{Remark}
\newtheorem{hypothesis}[definition]{Hypothesis}
%
\hypersetup{
    colorlinks,
    citecolor=black,
    filecolor=black,
    linkcolor=black,
    urlcolor=black
}
%
\title{...}
\author{...}
\date{...}
%
\begin{document}
%
\maketitle
%
\begin{abstract}
    ...
\end{abstract}
%
\tableofcontents
%
\section{Introduction}
...
\end{verbatim}


\newpage

\chapter*{TikZ}

\section*{Figures}

\begin{figure}[h]
    \centering
    \begin{subfigure}[b]{0.45\textwidth}
        \centering
        \begin{tikzpicture}
            \node[circle, draw] at (0,0){};
        \end{tikzpicture}
        \caption{First figure}
    \end{subfigure}
    \hfill
    \begin{subfigure}[b]{0.45\textwidth}
        \centering
        \begin{tikzpicture}
            \node[circle, draw] at (2,0){};
        \end{tikzpicture}
        \caption{Second figure}
    \end{subfigure}
    \caption{Two TikZ figures side by side}
\end{figure}

\begin{verbatim}
  \usepackage{subcaption}

  \begin{figure}[h]
    \centering
    \begin{subfigure}[b]{0.45\textwidth}
        \centering
        \begin{tikzpicture}
            \node[circle, draw] at (0,0){};
        \end{tikzpicture}
        \caption{First figure}
    \end{subfigure}
    \hfill
    \begin{subfigure}[b]{0.45\textwidth}
        \centering
        \begin{tikzpicture}
            \node[circle, draw] at (2,0){};
        \end{tikzpicture}
        \caption{Second figure}
    \end{subfigure}
    \caption{Two TikZ figures side by side}
  \end{figure}
\end{verbatim}

\begin{figure}[h]
    \centering
    \captionsetup{labelformat=empty}
    \begin{subfigure}[b]{0.45\textwidth}
        \centering
        \captionsetup{labelformat=empty}
        \begin{tikzpicture}
            \node[circle, draw] at (0,0){};
        \end{tikzpicture}
        \caption{First figure}
    \end{subfigure}
    \hfill
    \begin{subfigure}[b]{0.45\textwidth}
        \centering
        \captionsetup{labelformat=empty}
        \begin{tikzpicture}
            \node[circle, draw] at (2,0){};
        \end{tikzpicture}
        \caption{Second figure}
    \end{subfigure}
    \caption{Two TikZ figures side by side}
\end{figure}

\begin{verbatim}
  \usepackage{subcaption}

  \begin{figure}[h]
      \centering
      \captionsetup{labelformat=empty}
      \begin{subfigure}[b]{0.45\textwidth}
          \centering
          \captionsetup{labelformat=empty}
          \begin{tikzpicture}
              \node[circle, draw] at (0,0){};
          \end{tikzpicture}
          \caption{First figure}
      \end{subfigure}
      \hfill
      \begin{subfigure}[b]{0.45\textwidth}
          \centering
          \captionsetup{labelformat=empty}
          \begin{tikzpicture}
              \node[circle, draw] at (2,0){};
          \end{tikzpicture}
          \caption{Second figure}
      \end{subfigure}
      \caption{Two TikZ figures side by side}
  \end{figure}
\end{verbatim}

\textbf{Use} 

\begin{verbatim} 
  \caption*{\textbf{Figure:} A simple diagram.} 
\end{verbatim} 

\textbf{to remove the entry from the list of figures.}

\section*{Axis}

\begin{center}
  \begin{tikzpicture}[x=1cm,y=1cm]
    
    \draw[->] (-1,0) -- (4,0)   node[right] {$x$};
    \draw[->] (0,-1) -- (0,4)   node[above] {$y$};

  \end{tikzpicture}
\end{center}

\begin{verbatim}
\begin{center}
  \begin{tikzpicture}[x=1cm,y=1cm]
    
    \draw[->] (-1,0) -- (4,0)   node[right] {$x$};
    \draw[->] (0,-1) -- (0,4)   node[above] {$y$};

  \end{tikzpicture}
\end{center}
\end{verbatim}

\begin{center}
  \begin{tikzpicture}[x=1cm,y=1cm]
    
    \draw[->] (-1,0) -- (4,0)   node[right] {$x$};
    \draw[->] (0,-1) -- (0,4)   node[above] {$y$};

    \draw (-1,0) -- (4,1) node[anchor=west]{$\ell_1$};
    \draw (0,0) -- (4,2) node[anchor=west]{$\ell_2$};
    \draw (1,0) -- (4,3) node[anchor=west]{$\ell_3$};
    \draw (2,0) -- (4,4) node[anchor=west]{$\ell_4$};
  \end{tikzpicture}
\end{center}

\begin{verbatim}
\begin{center}
  \begin{tikzpicture}[x=1cm,y=1cm]
    
    \draw[->] (-1,0) -- (4,0)   node[right] {$x$};
    \draw[->] (0,-1) -- (0,4)   node[above] {$y$};

    \draw (-1,0) -- (4,1) node[anchor=west]{$\ell_1$};
    \draw (0,0) -- (4,2) node[anchor=west]{$\ell_2$};
    \draw (1,0) -- (4,3) node[anchor=west]{$\ell_3$};
    \draw (2,0) -- (4,4) node[anchor=west]{$\ell_4$};
  \end{tikzpicture}
\end{center}
\end{verbatim}

\begin{center}
  \begin{tikzpicture}
    \begin{axis}[
      x=1.0cm,y=1.0cm,
      axis lines=middle,
      xmajorgrids=true,
      ymajorgrids=true,
      xmin=-1.0,
      xmax=2.0,
      ymin=-1.0,
      ymax=2.0,
      xtick={-1.0,0.0,...,2.0},
      ytick={-1.0,0.0,...,2.0},
    ]
      \clip(-1.0,-1.0) rectangle (2.0,2.0);
      \fill[line width=4.pt,color=red,fill=red,pattern=north east lines,pattern color=red] (0.,0.) -- (0.,1.) -- (1.,1.) -- (1.,0.) -- cycle;
    \end{axis}
  \end{tikzpicture}
\end{center}

\begin{verbatim}
\begin{center}
  \begin{tikzpicture}
    \begin{axis}[
      x=1.0cm,y=1.0cm,
      axis lines=middle,
      xmajorgrids=true,
      ymajorgrids=true,
      xmin=-1.0,
      xmax=2.0,
      ymin=-1.0,
      ymax=2.0,
      xtick={-1.0,0.0,...,2.0},
      ytick={-1.0,0.0,...,2.0},
    ]
      \clip(-1.0,-1.0) rectangle (2.0,2.0);
      \fill[line width=4.pt,color=red,fill=red,pattern=north east lines,pattern color=red] (0.,0.) -- (0.,1.) -- (1.,1.) -- (1.,0.) -- cycle;
    \end{axis}
  \end{tikzpicture}
\end{center}
\end{verbatim}

\begin{center}
    \begin{tikzpicture}
        \begin{axis}[
            x=0.3cm,y=0.3cm,
            axis lines=middle,
            xmin=-4.0,
            xmax=40.0,
            ymin=-3.0,
            ymax=23.0,
            xtick={-4.0,-2.0,...,40.0},
            ytick={-2.0,0.0,...,22.0},
        ]
            \clip(-4.,-3.) rectangle (40.,23.);
            \fill[line width=8.pt,color=red,fill=red,pattern=north east lines,pattern color=red] (3.,10.) -- (5.,14.) -- (12.,14.) -- (16.,11.) -- (9.,4.) -- cycle;
            \draw [line width=1.pt,domain=-12.:40.] plot(\x,{(--13.-1.*\x)/1.});
            \draw [line width=1.pt,domain=-12.:40.] plot(\x,{(--5.-1.*\x)/-1.});
            \draw [line width=1.pt,domain=-12.:40.] plot(\x,{(--4.--2.*\x)/1.});
            \draw [line width=1.pt,domain=-12.:40.] plot(\x,{(--92.-3.*\x)/4.});
            \draw [line width=1.pt,domain=-12.:40.] plot(\x,{(--14.-0.*\x)/1.});
            \begin{scriptsize}
                \draw [fill=black] (-1.,14.) circle (2.5pt);
                \draw [fill=black] (3.,10.) circle (2.5pt);
                \draw [fill=black] (5.,14.) circle (2.5pt);
                \draw [fill=black] (6.909090909090909,17.818181818181817) circle (2.5pt);
                \draw [fill=black] (12.,14.) circle (2.5pt);
                \draw [fill=black] (19.,14.) circle (2.5pt);
                \draw [fill=black] (16.,11.) circle (2.5pt);
                \draw [fill=black] (9.,4.) circle (2.5pt);
                \draw [fill=black] (-2.,0.) circle (2.5pt);
                \draw [fill=black] (-2.,0.) circle (2.5pt);
                \draw [fill=black] (0.,0.) circle (2.5pt);
                \draw [fill=black] (0.,4.) circle (2.5pt);
                \draw [fill=black] (5.,0.) circle (2.5pt);
                \draw [fill=black] (13.,0.) circle (2.5pt);
                \draw [fill=black] (30.666666666666668,0.) circle (2.5pt);
            \end{scriptsize}
        \end{axis}
    \end{tikzpicture}
\end{center}

\begin{verbatim}
  \begin{center}
    \begin{tikzpicture}
        \begin{axis}[
            x=0.3cm,y=0.3cm,
            axis lines=middle,
            xmin=-4.0,
            xmax=40.0,
            ymin=-3.0,
            ymax=23.0,
            xtick={-4.0,-2.0,...,40.0},
            ytick={-2.0,0.0,...,22.0},
        ]
            \clip(-4.,-3.) rectangle (40.,23.);
            \fill[
              line width=8.pt,
              color=red,
              fill=red,
              pattern=north east lines,
              pattern color=red
            ] (3.,10.) -- (5.,14.) -- (12.,14.) -- (16.,11.) -- (9.,4.) -- cycle;
            \draw [line width=1.pt,domain=-12.:40.] plot(\x,{(--13.-1.*\x)/1.});
            \draw [line width=1.pt,domain=-12.:40.] plot(\x,{(--5.-1.*\x)/-1.});
            \draw [line width=1.pt,domain=-12.:40.] plot(\x,{(--4.--2.*\x)/1.});
            \draw [line width=1.pt,domain=-12.:40.] plot(\x,{(--92.-3.*\x)/4.});
            \draw [line width=1.pt,domain=-12.:40.] plot(\x,{(--14.-0.*\x)/1.});
            \begin{scriptsize}
                \draw [fill=black] (-1.,14.) circle (2.5pt);
                \draw [fill=black] (3.,10.) circle (2.5pt);
                \draw [fill=black] (5.,14.) circle (2.5pt);
                \draw [fill=black] (6.909090909090909,17.818181818181817) circle (2.5pt);
                \draw [fill=black] (12.,14.) circle (2.5pt);
                \draw [fill=black] (19.,14.) circle (2.5pt);
                \draw [fill=black] (16.,11.) circle (2.5pt);
                \draw [fill=black] (9.,4.) circle (2.5pt);
                \draw [fill=black] (-2.,0.) circle (2.5pt);
                \draw [fill=black] (-2.,0.) circle (2.5pt);
                \draw [fill=black] (0.,0.) circle (2.5pt);
                \draw [fill=black] (0.,4.) circle (2.5pt);
                \draw [fill=black] (5.,0.) circle (2.5pt);
                \draw [fill=black] (13.,0.) circle (2.5pt);
                \draw [fill=black] (30.666666666666668,0.) circle (2.5pt);
            \end{scriptsize}
        \end{axis}
    \end{tikzpicture}
\end{center}
\end{verbatim}

\begin{center}
  \begin{tikzpicture}[x=0.03cm,y=0.015cm]

    \fill[yellow!50] (0,0) -- (100,0) -- (40,240) -- (0,300) -- cycle;

    % Axes
    \draw[->] (-5,0) -- (220,0) node[below right] {$x$};
    \draw[->] (0,-5) -- (0,420) node[above left] {$y$};

    \foreach \x/\lab in {40/40,100/100,200/200}{
      \draw (\x,0) -- (\x,-5) node[below] {\lab};
    }

    \foreach \y/\lab in {200/200,240/240,400/400}{
      \draw (0,\y) -- (-5,\y) node[left] {\lab};
    }

    \draw[thick] (0,400) -- (100,0)
      node[pos=0.42,above left,sloped] {$40\,x + 10\,y = 4000$};

    \draw[thick] (0,300) -- (200,0)
      node[pos=0.65,above,sloped] {$30\,x + 20\,y = 6000$};

    \draw[thick,red,yshift=-10,xshift=-10] (0,300) -- (150,0)
      node[pos=0.75,above,sloped] {$400\,x + 200\,y = \lambda$};

    \fill (40,240) circle [radius=1.2];
    \draw[densely dotted] (40,0) -- (40,240);
    \draw[densely dotted] (0,240) -- (40,240);
  \end{tikzpicture}
\end{center}

\begin{verbatim}
  \begin{tikzpicture}[x=0.03cm,y=0.015cm]

    \fill[yellow!50] (0,0) -- (100,0) -- (40,240) -- (0,300) -- cycle;

    % Axes
    \draw[->] (-5,0) -- (220,0) node[below right] {$x$};
    \draw[->] (0,-5) -- (0,420) node[above left] {$y$};

    \foreach \x/\lab in {40/40,100/100,200/200}{
      \draw (\x,0) -- (\x,-5) node[below] {\lab};
    }

    \foreach \y/\lab in {200/200,240/240,400/400}{
      \draw (0,\y) -- (-5,\y) node[left] {\lab};
    }

    \draw[thick] (0,400) -- (100,0)
      node[pos=0.42,above left,sloped] {$40\,x + 10\,y = 4000$};

    \draw[thick] (0,300) -- (200,0)
      node[pos=0.65,above,sloped] {$30\,x + 20\,y = 6000$};

    \draw[thick,red,yshift=-10,xshift=-10] (0,300) -- (150,0)
      node[pos=0.75,above,sloped] {$400\,x + 200\,y = \lambda$};

    \fill (40,240) circle [radius=1.2];
    \draw[densely dotted] (40,0) -- (40,240);
    \draw[densely dotted] (0,240) -- (40,240);
  \end{tikzpicture}
\end{verbatim}

\end{document}