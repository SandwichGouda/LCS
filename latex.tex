\documentclass{report}



\usepackage[french]{babel}

\usepackage{parskip}
\usepackage{geometry}

\usepackage{amsmath}
\usepackage{amssymb}
\usepackage{amsfonts}
\usepackage{dsfont}
\usepackage{stmaryrd}
\usepackage{esint}

\usepackage[utf8]{inputenc}
\usepackage[T1]{fontenc}
\usepackage{lmodern}

\usepackage{color}
\usepackage{mathdots}
\usepackage{ulem}
\usepackage{cancel}

\usepackage{accents}

\usepackage{url}

%%%% PEIGNE FIN TUTO SDZ PR AVOIR TOUT 
%%%% S gothique (et autres) mathfrak ? (doivent etre inclus dans la table polices maths (A arrongi, L endomorph/laplace tho)
%%%% tA (transposée) expôsant à gauche ?
%%%% a° avec un point au dessus comme l'intérieur ? 
%%%% layout thing ; llbrackets piur ensmeble des entiers ?




\begin{document}

%\textwidth = 690pt

\newgeometry{right = 0in, top = 1.2in}

\url{https://latexref.xyz}

\section*{Symboles} %%%% ------------------------------------------------- SYMBOLES ------------------------------------------------- %%%%

\begin{tabular}{ll}


Exposants $ x^{n} $
&
\texttt{
 x\string^\{n\}
}
\\


Racines $ \sqrt[n]{x} $
&
\texttt{
 \textbackslash{}sqrt[n]\{x\}
}
\\

Limites $ \lim_{x \to +\infty} \quad \displaystyle{\lim_{x \to \infty}} $
&
\texttt{
	\textbackslash{}lim\_\{x \textbackslash{}to +\textbackslash{}infty\} \quad \textbackslash{}displaystyle\{ ... \}
}
\\

$ \xrightarrow[dessous]{dessus} $
&
\texttt{
\textbackslash{}xrightarrow[dessous]\{dessus\}
}
\\

$ \displaystyle{f(x) \longrightarrow_{x \to +\infty}l} $
&
\\

Sommes $ \sum_{k=1}^{n} $
&
\texttt{
\textbackslash{}sum\_\{k=1\}\string^\{n\} \quad \textbackslash{}displaystyle\{ ... \}
}
\\


Produits $ \prod_{i=1}^{n} $
&
\texttt{
\textbackslash{}prod\_\{i=1\}\string^\{n\} \quad \textbackslash{}displaystyle\{ ... \}
}
\\

Union/inter. $ \bigcup_{i=1}^{n} \bigcap_{i=1}^{n} $
&
\texttt{
\textbackslash{}bigcup\_\{i=1\}\string^\{n\} \quad ...cap... \quad \textbackslash{}displaystyle\{ ... \}
}
\\

Bigoplus $ \bigoplus_{k=1}^{n} \bigotimes $
&
\texttt{
\textbackslash{}bigoplus\_\{i=1\}\string^\{n\} \quad \textbackslash{}displaystyle\{ ... \}
}
\\

Intégrales $ \int $
&
\texttt{
\textbackslash{}int
}
\\

Avec bornes $ \int_{a}^{b} blabladx $
&
\texttt{
\textbackslash{}int\_\{a\}\string^\{b\}
}
\\

Avec bornes hautes $ \int \limits_{a}^{b} blabladx $
&
\texttt{
\textbackslash{}int \textbackslash{}limits\_\{a\}\string^\{b\}
}
\\

Intégrales multiples $ \iint_{S} \iiint_{V} $
&
\texttt{
\textbackslash{}iint\_\{S\} \textbackslash{}iiint\_\{V\}
}
\\

Sur un contour fermé $ \oint $
&
\texttt{
\textbackslash{}oint
}
\\

Grandes parenthèses
&
\texttt{
\textbackslash{}left( \textbackslash{}right)
}
\\

\textbf{Autres grands trucs}
&
\texttt{
\textbackslash{}left\textbackslash{}lfloor \textbackslash{}right\textbackslash{}rfloor \textbackslash{}left\textbackslash{}langle \textbackslash{}right\textbackslash{}rangle
}
\\

Valeurs absolues $ |x| \quad \left| x \right| $
&
\texttt{
|x| \quad \textbackslash{}left| x \textbackslash{}right| 
}
\\


Pareil mais chiant $ \mathopen| x \mathclose| $   
&
\texttt{
\textbackslash{}mathopen| x \textbackslash{}mathclose|
}
\\

Egalités $ = \quad \triangleq \quad \ne \quad \equiv $
&
\texttt{
= \textbackslash{}triangleq \textbackslash{}ne ou \textbackslash{}neq \textbackslash{}equiv
}
\\

$ \approx \quad \simeq \quad \sim \quad \cong \quad \thicksim \quad \gtrsim \quad \lesssim $
&
\texttt{
\textbackslash{}approx \textbackslash{}simeq \textbackslash{}sim \textbackslash{}cong \textbackslash{}thicksim \textbackslash{}gtrsim \textbackslash{}lesssim
}
\\


Inégalités $ \leq \quad \geq \quad \leqslant \quad \geqslant $
&
\texttt{
\textbackslash{}leq \textbackslash{}geq \textbackslash{}leqslant \textbackslash{}geqslant
}
\\

$ \ll \quad \gg \quad \prec \quad \succ \quad \preceq \quad \succeq $
&
\texttt{
\textbackslash{}ll \textbackslash{}gg \textbackslash{}prec \textbackslash{}succ \textbackslash{}preceq \textbackslash{}succeq	
}
\\

Ensembles $ \in \quad \notin \quad \subset \quad \not\subset \quad \setminus $
&
\texttt{
\textbackslash{}in \textbackslash{}notin \textbackslash{}subset \textbackslash{}not\textbackslash{}subset \textbackslash{}setminus 
}
\\

$ \subseteq \quad \nsubseteq \quad \cap \quad \cup \quad \sqcup \quad \supset \quad \varnothing $
&
\texttt{
\textbackslash{}(n)subseteq \textbackslash{}c(a/u)p \textbackslash{}sqc(u/a)p \textbackslash{}supset \textbackslash{}varnothing
}
\\

Opérations
&
\\

$ \times \quad \div \quad g \circ f \quad \star \quad \pm \quad \mp \quad \ast $
&
\texttt{
\textbackslash{}times \textbackslash{}div \textbackslash{}circ \textbackslash{}star \textbackslash{}pm \textbackslash{}mp \textbackslash{}ast 
}
\\

$ \wedge \quad \vee \quad \oplus \quad \otimes \quad \ominus \quad \odot $
&
\texttt{
\textbackslash{}wedge \textbackslash{}vee \textbackslash{}oplus \textbackslash{}otimes \textbackslash{}ominus \textbackslash{}odot	
}
\\

Logique
&
\\

$ \forall \quad \exists \quad \exists! \quad \nexists \quad \neg \quad \lor \quad \land $
&
\texttt{
\textbackslash{}forall \textbackslash{}exists \textbackslash{}exists! \textbackslash{}nexists \textbackslash{}neg \textbackslash{}lor \textbackslash{}land
}
\\

$ \implies \quad \iff \quad \impliedby \quad \Rightarrow \quad \Leftrightarrow $
&
\texttt{
\textbackslash{}implies \textbackslash{}iff \textbackslash{}impliedby \textbackslash{}(Right/Leftright)arrow
}
\\

$ \mapsto \quad \longmapsto \quad \to \quad \longrightarrow $
&
\texttt{
\textbackslash{}mapsto \textbackslash{}longmapsto \textbackslash{}to \textbackslash{}longrightarrow
}
\\

Autres flèches
&
\\

$  \nearrow \quad \hookrightarrow \quad \leadsto \quad \rightleftharpoons $
&
\texttt{
\textbackslash{}nearrow \textbackslash{}hookrightarrow \textbackslash{}leadsto \textbackslash{}rightleftharpoons
}
\\

Crochets $ \lfloor \; \; \rfloor \quad \lceil \; \; \rceil \quad [\, \,] \quad \langle \quad \rangle $
&
\texttt{
\textbackslash{}lfloor \textbackslash{}rfloor \textbackslash{}rceil \textbackslash{}lceil [\textbackslash{}, \textbackslash{},] \textbackslash{}langle \textbackslash{}rangle 
}
\\

$\llbracket \quad \rrbracket$
&
\texttt{
\textbackslash{}llbracket \textbackslash{}rrbracket (\textbackslash{}usepackage\{stmaryrd\})
}
\\

Normes $ \lVert \, \cdot \, \rVert \quad |||\cdot||| \quad \| u \|$
&
\texttt{
\textbackslash{}lVert \textbackslash{}rVert \quad |||u||| \quad \textbackslash{}| u \textbackslash{}|
}
\\

\textit{ Ajouter \texttt{ \textbackslash{},} avec \texttt{ \textbackslash{}cdot } }
&
\textit{ et \texttt{ \textbackslash{}\!\! ;} avec des grosses $ \Sigma $ }
\\

Divers / jsp
&
\\

$ \mid \quad \nmid \quad \parallel \quad \bowtie \quad \dagger \quad \ddagger \quad \square	$
&
\texttt{
\textbackslash{}mid \textbackslash{}nmid \textbackslash{}parallel \textbackslash{}bowtie \textbackslash{}dagger \textbackslash{}ddagger \textbackslash{}square
}
\\

$ \triangleleft \quad \triangleright \quad \bigtriangleup \quad \bigtriangledown \quad \Delta \quad \nabla	$
&
\texttt{
\textbackslash{}(big)triangle(left/right) \textbackslash{}Delta \textbackslash{}nabla
}
\\

$ \perp \quad \top \quad \partial \quad \hbar \quad \ell \quad \Re \quad \Im $
&
\texttt{
\textbackslash{}perp \textbackslash{}top \textbackslash{}partial \textbackslash{}hbar \textbackslash{}ell \textbackslash{}Re \textbackslash{}Im
}
\\

$ \epsilon \quad \varepsilon \quad \phi \quad \varphi $
&
\texttt{
\textbackslash{}epsilon \textbackslash{}varepsilon \textbackslash{}phi \textbackslash{}varphi
}
\\


\end{tabular}	

\restoregeometry

\newpage


\section*{Lettres}  %%%% ------------------------------------------------- LETTRES ------------------------------------------------- %%%%


\begin{tabular}{ll}



$ \mathds{N}, \mathds{Z}, \mathds{Q}, \mathds{R}, \mathds{C} $
&
\texttt{
\textbackslash{}mathds\{X\}
}
\\

$ {\rm I\!N}, {\rm I\!R}  $
&
\texttt{
\{\textbackslash{}rm I\textbackslash{}\!\! !X\}
}
\\

$ \mathbb{N}, \mathbb{Z}, \mathbb{Q}, \mathbb{R}, \mathbb{C} $
&
\texttt{
\textbackslash{}mathbb\{X\}
}
\\	

$ \mathbf{N}, \mathbf{Z}, \mathbf{Q}, \mathbf{R}, \mathbf{C} $
&
\texttt{
\textbackslash{}mathbf\{X\}
}
\\

$ \mathcal{ABCDEFHKLMOPSUVW} $
&
\texttt{
\textbackslash{}mathcal\{ABC\}
}
\\

$ \mathfrak{ABCDEFGHIJKLMNOPQRSTUVWXYZ} $
&
\texttt{
\textbackslash{}mathfrak\{ABC\}
}
\\

\end{tabular}


\section*{Accentuations}  %%%% -------------------------------------------- ACCENTUATIONS -------------------------------------------- %%%%


\begin{tabular}{ll}


Accent circonflexe (texte) $ \string^ $
&
\texttt{
\textbackslash{}string\string^
}
\\


Accent circonflexe (accent) \^{a}
&
\texttt{
\textbackslash{}\string^\{a\}
}
\\

Accent circonflexe (accent)(maths) $ \hat{a} $
&
\texttt{
\textbackslash{}hat\{a\}
}
\\

Produit vectoriel / PPCM $ \wedge \vee $
&
\texttt{
\textbackslash{}wedge \textbackslash{}vee
}
\\

Angle genre $ \widehat{abc} $
&
\texttt{
\textbackslash{}widehat\{abc\}
}
\\

Accent aigu $ \acute{a} $
&
\texttt{
\textbackslash{}acute\{a\}
}
\\

(peut s'appliquer à un mot entier)
&
\\

Accent grave $ \grave{a} $
&
\texttt{
\textbackslash{}grave\{a\}
}
\\

Barre $ \bar{a} \quad \bar{10100}^{2} $ (largeur fixe)
&
\texttt{
\textbackslash{}bar\{a\}
}
\\

Overline $ \overline{a} \quad \overline{10100}^{2} $
&
\texttt{
\textbackslash{}overline\{a\}
}
\\

Underline $ \underline{a} \quad \underline{Z} = \underline{U}/\underline{I} $
&
\texttt{
\textbackslash{}underline\{a\}
}
\\

Overbrace $ \overbrace{abc} $
&
\texttt{
\textbackslash{}overbrace\{abc\}
}
\\

Underbrace $ \underbrace{abc} $
&
\texttt{
\textbackslash{}underbrace\{abc\}
}
\\

Overset $ \overset{a}{X} \quad \overset{\circ}{B} $
&
\texttt{
\textbackslash{}overset\{a\}\{X\}
}
\\

Underset $ \underset{a}{X} $
&
\texttt{
\textbackslash{}underset\{a\}\{X\}
}
\\

Text overbrace $ \overbrace{mainthing}^{overtext} $
&
\texttt{
\textbackslash{}overbrace\{mainthing\}\string^\{overtext\}
}
\\

Text underbrace $ \underbrace{mainthing}_{undertext} $
&
\texttt{
\textbackslash{}underbrace\{mainthing\}\_\{undertext\}
}
\\

Point $ \dot{x} $
&
\texttt{
\textbackslash{}dot\{x\}
}
\\

Point point $ \ddot{x} $
&
\texttt{
\textbackslash{}ddot\{x\}
}
\\

Tilde $ \tilde{u} $
&
\texttt{
\textbackslash{}tilde\{u\}
}
\\


Widetilde $ \widetilde{abc} $
&
\texttt{
\textbackslash{}widetide\{abc\}
}
\\

Vecteur $ \overrightarrow{v} \quad \overrightarrow{grad} $ 
&
\texttt{
\textbackslash{}overrightarrow\{grad\}
}
\\

Vecteur $ \vec{v} \quad \vec{grad} $  (moche)
&
\texttt{
\textbackslash{}vec\{v\}
}
\\

Produit scalaire
&
\\

$ \overrightarrow{u} \cdot \overrightarrow{v} $
&
\texttt{
\textbackslash{}cdot
}
\\

$ \overrightarrow{u} \cdotp \overrightarrow{v} $
&
\texttt{
\textbackslash{}cdotp
}
\\

$ \overrightarrow{u} \bullet \overrightarrow{v} $
&
\texttt{
\textbackslash{}bullet
}
\\

$ \accentset{\circ}{A} $
&
\texttt{
\textbackslash{}accentset\{\textbackslash{}circ\}\{I\} \quad \textbackslash{}usepackage\{accents\}
}
\\




%
&
\texttt{
%\textbackslash{}
}
\\

%
&
\texttt{
%\textbackslash{}
}
\\

\end{tabular}



\subsection*{Espacements} % \normalfont (+ alignement)

%\begin{align*}

\begin{tabular}{ll}

$ x^{2} \! + \! 3x \! + \! 2 $
& 
\texttt{ \textbackslash{}\!\! ! }
\\

$ x^{2}+3x+2 $ 
&
\texttt{ [rien] }
\\

$ x^{2} + 3x + 2 $ 
&
\texttt{ [espaces] }
\\

$ x^{2}\, +\, 3x\, +\, 2 $
&
\texttt{ \textbackslash{},}
\\

$ x^{2}\: +\: 3x\: +\: 2 $
&
\texttt{ \textbackslash{}\!\!\! :}
\\

$ x^{2}\; +\; 3x\; +\; 2 $
&
\texttt{ \textbackslash{}\!\! ;}
\\

$ x^{2}\ +\ 3x\ +\ 2 $
&
\texttt{ \textbackslash{}}
\\

$ x^{2}\quad +\quad 3x\quad +\quad 2 $
&
\texttt{ \textbackslash{}quad}
\\

$ x^{2}\qquad +\qquad 3x\qquad +\qquad 2 $
&
\texttt{ \textbackslash{}qquad}
\\

\end{tabular}

%\end{align*} (aligne devant les & (à ajouter du coup, comme point d'ancrage)

\subsection*{Fractions avec \normalfont \texttt{ \textbackslash{}frac\{\}\{\}  } }

\begin{center}
$ 1 + \frac{1}{1+\frac{1}{1+...}} $
\qquad \qquad
$ 1 + \frac{1}{1+\frac{1}{1+\frac{1}{1+\frac{1}{1+\frac{1}{1+...}}}}} $
\end{center}

\subsection*{Fractions avec \normalfont \texttt{ \textbackslash{}cfrac\{\}\{\}  } }

\begin{center}
$$ 1 + \cfrac{1}{1+\cfrac{1}{1+...}},  
\qquad\qquad
1 + \cfrac{1}{1+\cfrac{1}{1+\cfrac{1}{1+\cfrac{1}{1+\cfrac{1}{1+...}}}}} $$
\end{center}

\subsection*{Autres}

\subsubsection*{Fonction usuelles}

\begin{center}

$ \cos(x), \sin(x), \tan(x), \arccos(x), \arcsin(x), \arctan(x), \cosh(x), \sinh(x), $ \\ 
$ \tanh(x), \cosh^{-1}(x), \sinh^{-1}(x), \tanh^{-1}(x),  \exp(x), \ln(x), \log(x), \log_{b}(a) $ \\
$ \arg(x), \dim(x), \min(a,b), \max(a,b), \gcd(a,b) $ \\

\bigskip

$ cos(x), sin(x), tan(x), arccos(x), arcsin(x), arctan(x), cosh(x), sinh(x), $ \\
$ tanh(x), cosh^{-1}(x), sinh^{-1}(x), tanh^{-1}(x),  exp(x), ln(x), log(x), log_{b}(a) $ \\
$ arg(x), dim(x), min(a,b), max(a,b), gcd(a,b) $ \\

\end{center}

\textbf{Pour d'autres fonctions : }
\texttt{ \$ \textbackslash{}mathrm\{PGCD\} \$}

\textbf{Pour ajouter des trucs en dessous comme ça :}
$$ \sum_{\substack{(x,K) \, tq \\ x \in \Omega \\ K \subset I_{x}}} (-1)^{\mathrm{Card(K)}} $$ 
\begin{center}  \texttt{ \textbackslash{}sum\_\{ \textbackslash{}substack\{ ... \textbackslash{}\textbackslash{}
... \textbackslash{}\textbackslash{} ... \} \}} \\
\textbf{Ne pas oublier les brackets pour substack.}\end{center}

\newpage

\begin{center} 

$$ \iint_{S} \mu(x,y) dxdy $$

$$ \int \!\!\!\!\!\! \bigcirc \!\!\!\!\!\! \int_{\Sigma} $$

\texttt{
\$\$ \textbackslash{}int \textbackslash{}!\textbackslash{}!\textbackslash{}!\textbackslash{}!\textbackslash{}!\textbackslash{}!\textbackslash{}! \textbackslash{}bigcirc \textbackslash{}!\textbackslash{}!\textbackslash{}!\textbackslash{}!\textbackslash{}!\textbackslash{}!\textbackslash{}! \textbackslash{}int\_\{Sigma\} \$\$
}

$$ \left\lVert \: \sum_{i=1}^{n} \lambda_i e_i \: \right\rVert $$ 

\texttt{
\textbackslash{}left\textbackslash{}lVert \textbackslash{}\!\!\!: \textbackslash{}sum ... \textbackslash{}\!\!\!: \textbackslash{}right\textbackslash{}rVert
}


$$ f(x) \to \ell $$ 
\texttt{ \$ f(x) \textbackslash{}to \textbackslash{}ell \$} 

$$ f(x) \rlap{\hskip 3mm $/$} \longrightarrow \ell $$ 
\texttt{ \$ f(x) \textbackslash{}rlap\{\textbackslash{}hskip 3mm \$/\$\} \textbackslash{}longrightarrow \textbackslash{}ell \$}

\end{center}

\bigskip
\bigskip
\bigskip

\Large \textbf{Changer la numérotation des part, chapter, section, etc}

\normalsize

Déjà, dans l'ordre

\texttt{ \textbackslash{}part \\ \textbackslash{}chapter \\ \textbackslash{}section \\ \textbackslash{}subsection \\ \textbackslash{}subsubsection \\ \textbackslash{}paragraph \\ \textbackslash{}subparagraph }

\chapter{Chapter}

\section{Section}

\subsection{Subsection}

\subsubsection{Subsubsection}

\paragraph{Paragraph}

\subparagraph{Subparagraph}

\paragraph{}

\paragraph{}

Et ensuite pour renommer

\texttt{ 
\textbackslash{}renewcommand\textbackslash{}thepart\{\textbackslash{}arabic\{part\}\} \\
\textbackslash{}renewcommand\textbackslash{}thesection\{\textbackslash{}arabic\{section\}\} \\
\textbackslash{}renewcommand\textbackslash{}thesubsection\{\textbackslash{}arabic\{subsection\}\} \\
\textbackslash{}arabic : 1, 2, 3, ... \\
\textbackslash{}alph : a, b, c, ... \\
\textbackslash{}Alph : A, B, C, ... \\
\textbackslash{}roman : i, ii, iii, ... \\
\textbackslash{}Roman : I, II, III, ...
}	

\newpage

\textbf{Systèmes d'équations}

\begin{align*}
x + y + z &= a\\
x - y &= b\\
z &= c
\end{align*}

\begin{center}
\texttt{
\textbackslash{}begin\{align*\} .. \&= .. \textbackslash{}\textbackslash{} .. \&= .. \textbackslash{}end\{align*\}
}\\
\end{center}

\begin{eqnarray*}
x + y + z &=& a\\
x - y &=& b\\
z &=& c
\end{eqnarray*}

\begin{center}
\texttt{
\textbackslash{}begin\{eqnarray*\} .. \&=\& .. \textbackslash{}\textbackslash{} .. \&=\& .. \textbackslash{}end\{eqnarray*\}
}\\
\end{center}

\textbf{Une seule esperluette pour align, deux pour  array.}\\
Enlever les astérisques numérote les équations. 

$$
\left\{
\begin{array}{r c l}
x + y + z &=& a\\
x - y &=& b\\
z &=& c
\end{array}
\right.
$$

\texttt{
\$\$ \\
\textbackslash{}left\textbackslash{}\{ \\
\textbackslash{}begin\{array\}\{r c l\} \\
... \&=\& ... \\
... \&=\& ... \\
\textbackslash{}end\{array\} \\
\textbackslash{}right. \\
\$\$
} \\
Le point représente une \textbf{absence de délimiteurs.} \\
(on pourrait choisir de refermer l'accolade à droite).
\bigskip
\textbf{Autres délimiteurs :}





$$
\left(
\begin{array}{r c l}
x + y + z &=& a\\
y - z &=& b\\
z &=& c 
\end{array}
\right)
\left[
\begin{array}{r c l}
x + y + z &=& a\\
y - z &=& b\\
z &=& c 	
\end{array}
\right]
$$
$$
\left|
\begin{array}{r c l}
x + y + z &=& a\\
x - y &=& b\\
z &=& c
\end{array}
\right|
\quad
\left\|
\begin{array}{r c l}
x + y + z &=& a\\
x - y &=& b\\
z &=& c
\end{array}
\right\|
$$
\bigskip
\begin{center}
\texttt{ \textbackslash{}left( \qquad \textbackslash{}left[ \qquad \textbackslash{}left| \qquad \textbackslash{}left\textbackslash{}| }
\end{center}

\newpage

\subsubsection*{Les matrices}

Avec l'environnement \texttt{ array}  (et des \textbackslash{}quad) :
$$
\left[
\begin{array}{r c l}
1 \quad 2 \quad 3 \\
4 \quad 5 \quad 6 \\
7 \quad 8 \quad 9
\end{array}
\right] $$

\texttt{
\$\$ \\
\textbackslash{}left[ \\
\textbackslash{}begin\{array\}\{r c l\} \\
1 \textbackslash{}quad 2 \textbackslash{}quad 3 \textbackslash{}\textbackslash{} \\
4 \textbackslash{}quad 5 \textbackslash{}quad 6 \textbackslash{}\textbackslash{} \\
7 \textbackslash{}quad 8 \textbackslash{}quad 9 \\
\textbackslash{}end\{array\} \\
\textbackslash{}right] \\
\$\$ 
}

Avec \texttt{ matrix, pmatrix, bmatrix, vmatrix, 	Bmatrix, Vmatrix} :

$
\begin{matrix}
1 & 2 & 3 \\
4 & 5 & 6 \\
7 & 8 & 9
\end{matrix}
$
\quad
$
\begin{pmatrix}
1 & 2 & 3 \\
4 & 5 & 6 \\
7 & 8 & 9
\end{pmatrix}
$
\quad
$
\begin{bmatrix}
1 & 2 & 3 \\
4 & 5 & 6 \\
7 & 8 & 9
\end{bmatrix}
$
\quad
$
\begin{vmatrix}
1 & 2 & 3 \\
4 & 5 & 6 \\
7 & 8 & 9
\end{vmatrix}
$
\quad
$
\begin{Bmatrix}
1 & 2 & 3 \\
4 & 5 & 6 \\
7 & 8 & 9
\end{Bmatrix}
$
\quad
$
\begin{Vmatrix}
1 & 2 & 3 \\
4 & 5 & 6 \\
7 & 8 & 9
\end{Vmatrix}
$

\texttt{
\$\$ \\
\textbackslash{}begin\{matrix\} \\
.. \& .. \& .. \textbackslash{}\textbackslash{} \\
.. \& .. \& .. \textbackslash{}\textbackslash{} \\
.. \& .. \& .. \\
\textbackslash{}end\{matrix\} \\
\$\$
} 
\bigskip

\textbf{Pour les mettres les unes à côté des autres :} utiliser des \textbf{monodollars}, une paire par matrice, éventuellement séparés par des \texttt{ \textbackslash{}quad} comme ici.

\textbf{Technique ultime utilisée par Benhamou Sensei :}

$$ \left( \: \begin{matrix} 1 & 2 & 3 \\ 4 & 5 & 6 \\ 7 & 8 & 9 \end{matrix} \: \right) $$

Entourer l'environnement \texttt{ matrix} par \texttt{ \textbackslash{}left( \textbackslash{}\!\!\!:} et 
\texttt{ \textbackslash{}\!\!\!: \textbackslash{}right) }.

\textbf{Pointillés :}

\begin{tabular}{lllll}
$ \cdots $ & $ \ldots $ & $ \vdots $ & $ \ddots $ & $ \iddots $ \\
\texttt{ \textbackslash{}cdots } & \texttt{ \textbackslash{}ldots } & \texttt{ \textbackslash{}vdots } & \texttt{ \textbackslash{}ddots} & \texttt{ \textbackslash{}iddots} (\texttt{ \textbackslash{}usepackage\{mathdots\}})
\end{tabular}

La commande \texttt{ \textbackslash{}phantom} permet de gérer les alignements et le centrage des nombres dans chaque case.


%$
%\begin{bmatrix}
%   1 & 12345 & 3 \\
%   94 & 5 & -6 \\
%   7 & 8 & 9 
%\end{bmatrix}
%$
%\quad
%$
%\begin{bmatrix}
%   \phantom{9}1 & 12345 & \phantom{-}3 \\
%   94 & \phantom{1234}5 & -6 \\
%   \phantom{9}7 & \phantom{1234}8 & \phantom{-}9 
%\end{bmatrix}
%$

\subsection*{Matrices et applications}

Faire une belle application (Aymeric sensei no jutsu)

$$
\begin{array}{ccccc} 
\phi & : & \mathbb{N}^* & \to & \mathbb{N} \\ 
& & n & \mapsto & Card \left\{ k \in |[1,n]|, \, pgcd(k,n) = 1 \right\} \\ 
\end{array}
$$


\texttt{
\$ \textbackslash{}begin\{array\}\{ccccc\}  \\
f \& : \& \{\} \& \textbackslash{}to \& \{\} \textbackslash{}\textbackslash{}  \\
\& \& x \& \textbackslash{}mapsto \& ... \textbackslash{}\textbackslash{}  \\
\textbackslash{}end\{array\} \$
}

\textbf{On retiendra} : 
\texttt{ \textbackslash{}begin\{array\}\{ccccc\} } \\
Et mettre des esperluettes entre chaque truc, deux au début de la deuxième ligne (pour aligner $f$)
et pas en fin de ligne

Faire des belles matrices (Aymeric sensei no jutsu)

$$
\begin{bmatrix} 
a & b & c \\
d & e & f \\
g & h & i
\end{bmatrix}
$$

\texttt{
\$ \textbackslash{}begin\{bmatrix\}  \\
a \& b \& c \textbackslash{}\textbackslash{}  \\
a \& b \& c \textbackslash{}\textbackslash{}  \\
a \& b \& c \textbackslash{}\textbackslash{}  \\
\textbackslash{}end\{bmatrix\} \$
}

\textbf{On retiendra} : \texttt{ array bmatrix }, \\
esperluettes entre les objets, \\
et on revient à la ligne avec \texttt{ \textbackslash{}\textbackslash{} } 

%\textbackslash{} begin\{array\}\{ccccc\} 
%f \& : \& \{\} \& \to \& ... \textbackslash \textbackslash
%\& \& x \mapsto \& ...\textbackslash \textbackslash
%\textbackslash end\{array\}

%}

Il faut deux esperluettes au début de la deuxième ligne \\
pour que le f soit un peu décalé vers la gauche \\
Et sinon, un entre chaque truc, sauf en fin de ligne (y'a plus rien à aligner) 

Matrices par blocs

$$
\left(
\begin{array}{c|c}
A & B\\
\hline
C & D
\end{array}
\right)
$$

\texttt{
\$ \textbackslash{}left( \textbackslash{}begin\{array\}\{c|c\} \\
A \& B \textbackslash{}\textbackslash{} \\
\textbackslash{}hline \\
C \& D \\
\textbackslash{}end\{array\} \textbackslash{}right) \$
} Attention aux \textbackslash{}\textbackslash{}

%%%%%%%%% POINTILLES MATRICES   ---------    https://tex.stackexchange.com/questions/32217/3-dots-in-matrix/32221 %%%%%%%

\subsection*{Autres}

\textbf{Saut de ligne} : \texttt{ \textbackslash{}bigskip} \quad (ne \textit{pas} mettre en fin de ligne) \\
\textbf{\textcolor{red}{Couleurs :}} \texttt{ \textbackslash{}textcolor\{couleur\}\{texte\}  (\textbackslash{}usepackage\{color\})}\\
Pour afficher \texttt{ \textbackslash{}\!\! !}, il faut ajouter \texttt{ \textbackslash{}\!\! !\textbackslash{}\!\! !} entre \texttt{ \textbackslash{}} et \texttt{ !}. \\
Pour afficher \texttt{ \textbackslash{}\!\!\!:}, il faut ajouter \texttt{ \textbackslash{}\!\! !\textbackslash{}\!\! !\textbackslash{}\!\! !} entre \texttt{ \textbackslash{}} et \texttt{ :}. \\
Pour élargir le corps du texte : \texttt{ \textbackslash{}usepackage\{geometry\}} \quad \texttt{ \textbackslash{}textwidth = 426pt} \textbf{(non)} \\
Pour afficher certains caractères hors du math mode : \texttt{ \textbackslash{}textbullet \textbackslash{}dagger \textbackslash{}textasteriskcentered (p3 de la Big Great List)} \\
Pour encadrer : \texttt{ \textbackslash{}fbox\{...\}} ; remettre impérativement des dollars à l'intérieur si c'est dans une équation. \\
Lettre grecques en math mode \textit{uniquement}.\\
\texttt{ \textbackslash{}displaystyle\{\} :} mettre les dollars à \textbf{l'extérieur}. 

\subsubsection*{Remarques très anecdotiques}

Il y a  \texttt{ :} et \texttt{ colon} \\
Autres façons de faire l'ensemble vide : \texttt{ \textbackslash{}emptyset \textbackslash{}o } \\
$ \diamond $ \quad \texttt{ \textbackslash{}diamond} $ \surd $   \texttt{ \textbackslash{}surd}

\texttt{ \textbackslash{}Arrowvert}


\subsubsection*{Import de packages}

Belles lettres : \\
\texttt{
\textbackslash{}usepackage[utf8]\{inputenc\} \\
\textbackslash{}usepackage[T1]\{fontenc\} \\
\textbackslash{}usepackage\{lmodern\}
} \\
Français : \texttt{ \textbackslash{}usepackage[french]\{babel\} } \\
Quelques symboles de maths : \texttt{ \textbackslash{}usepackage\{amsmath\} } \\
Plus de symboles de maths : \texttt{ \textbackslash{}usepackage\{amssymb\} } \\
Belles lettres de maths : \texttt{ \textbackslash{}usepackage\{dsfont\} } (sans s) \\
Quelques polices : \texttt{ \textbackslash{}usepackage\{amsfonts\} } (avec s)\\
En particulier, cette police-la : \texttt{ \textbackslash texttt \{ ... \}}\\

\newpage

\section*{Commandes}

\section*{Packages}

\subsection*{\texttt{graphicx}}

\texttt{\textbackslash usepackage\{graphicx\} \\ \textbackslash includegraphics[scale=0.5]\{img.jpg\}}


\end{document}\